\documentclass{article}
\usepackage[utf8]{inputenc}
\usepackage[T1]{fontenc}
\usepackage[francais]{babel}
\title{Etat de l'art : Readers RSS}
\author{Anas ALAOUI M'DARHRI\\
Romain BRESSAN}

\begin{document}
\maketitle
\section{Principaux Readers RSS}
\subsection{Présentation des principaux readers}
Après une brève recherche sur le net, nous avons dégagé une dizaine de lecteurs de flux RSS. Certains d'entre eux ne sont nativement pas des \textit{RSS Reades}, mais disposent d'extensions permettant d'ajouter cette fonctionnalité. \\
Les principaux types de softwares disposant d'une lecture de flux RSS sont : 
\begin{itemize}
    \item Les agrégateurs de flux RSS 
    \item Les lecteurs de Mails 
    \item Les Applications de Musiquitem Les Navigateurs Web
\end{itemize}
L'explication à cela est simple : Dans le cas des lecteurs de Mails, le système fonctionne d'une manière analogue à la lecture de flux RSS, c'est pourquoi certains clients mail incorporent cette fonctionnalité (\textit{Mozilla Thunderbird} par example). \\
Dans le cas des applications de musique, le parallèle est moins évident. Cependant, certains d'entre eux gèrent nativement les podcasts audios ; Et donc incorporent un lecteur de flux RSS afin de récupérer directement les podcasts audios écoutés par l'utilisateur. \\
Enfin, les navigateurs web sont forcément capables de lire du XML, et donc de mettre en page et d'afficher des flux RSS. Certains navigateurs ont d'ailleurs incorporés la fonctionnalité d'agrégateur de flux, et offrent des services de gestion des flux RSS.\\
\subsection{AnalyseForces/Faiblesses}
Voici une liste non-exhaustive des différentes applications disposant d'une lecture de flux RSS, accompagnée d'une liste de leur forces/faiblesse.
\section{MetaMoteurs RSS}
\end{document}
